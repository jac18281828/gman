% LaTex 2e
% GMAN Manual
% Author: John Cairns
% DAte:   March 06, 1999
%
%


\documentclass{book}	% maybe

\title{GNU GMAN\\
Photorealistic Rendering Tool}
\author { John Cairns }
\date   { March 06, 1999 }

\def\gman{\bf GMAN \rm}
\def\gmansl{\bf GMANSL \rm}
\def\text#1{\bf\tt #1 \rm}
\def\instruction#1#2#3{\bf #1 \hfil\rm #2 \hfill \hbox{#3}\\}

\begin {document}

\maketitle
\eject
\vfill
\vbox{
\hsize 3in
\bf About the Author\rm\\
John Cairns is a computer nerd somewhere in Boulder, CO.   He finds
solace in knowing that there are other nerds out there who actually,
read this stuff, and perhap even find this program useful.   At least,
now he doesn't have to wait for Pixar to lower their prices.
}

\tableofcontents
\chapter{Introduction}

\begin{em}
'What the hell am I doing?'
\vskip .5 in
\end{em}

This is a fairly large program.  For my own personal benefit I still
consider the project to be a small one.   In fact, I never would have
'picked up the pen and began writing' if I thought this was a large
project.   There are a couple of important ways I made the project seem
smaller.  First, I borrowed the interface, file format, and shading
language from a known, trusted, and established source, Pixar's
RenderMan Specification\footnote{By the way, PhotoRealistic RenderMan, 
and Pixar are trademarks of Pixar Animation Studios, Inc.   
I am not affiliated with Pixar in any way, nor is this software.   
I wrote all the code from scratch, based on literature, 
and my experience as a programmer.}.
This helped make the job of writing a shading system a little
smaller, since I could minimize design time and focus on
\em algorithmic beauty.\rm   I also owe a huge thanks to Mark and Alan
Watt and their books \em 3D Computer Graphics,\rm and Advanced
Animation and Rendering Techniques\rm without which, this project
would have been untenable.

\section{About GMAN}
This library is intended to provide a high-quality rendering solution,
i.e. support for\footnote{Thanks to Larry Gritz of Pixar <lg@pixar.com>
for this list.}:

\begin{enumerate}
\item{High-end geometric types, such as trimmed curves, NURBS and
      Catmull-Clark subdivision surfaces.}
\item{Robust spatial and temporal antialiasing, 
      including motion-blurred geometric deformations\footnote{Whoa!
That's a mouthful.}
\item{Programmable shading of surfaces, lights, and volumes.}
\item{Displacement shading.}
\item{The ability to use large numbers (e.g. hundreds or thousands)
      of large textures without memory consumption growing appreciably.}
\item{Fine control over quantization, filtering and reconstruction of
      images.}
\item{Variable quality in rendering algorithims, providing fine
control over shading time versus shading quality.}
\end{enumerate}

These features are all required by the RenderMan standard, so choosing
RenderMan as a starting point was a good idea.   

\section {Free Software}

This library is free software.  Meaning that the source code is freely
avaiable.  \gman is distributed under the GNU General Public Library
License, available as part of the \gman distribution, or from \text{
http://www.fsf.org. }

\section{Downloading}

The current version of \gman is available at \text
{ http://www.2ad.com/gman. }
This Library is written in portable C and C++ and should be portable to
most\footnote{A good, modern C++ compiler such as GCC 2.8.0, is
required to compile \gman.}
major platforms.

\chapter{Library Design}
\begin{em}
'Enough nonsense lets get started!'
\end{em}
\section{Overview}

GMAN supports three interfaces for it's scene description language, C,
C++, and RIB.  The C interface is provided in the \text { ri.h }
header file.  The C++ interface is provided through the \text{ ripp.h }
header file, in the \text{ Ri } base class.  The C interface is
built out of the C++ Interface by providing a C wrapper that
instantiates a single instance of \text{ RiImpl, } the RenderMan
implementation class.

\subsection{Object Management}
\gman stores all objects defined with a \text{ RiWorldBegin,
RiWorldEnd } pair in an object database called an \text
{ ObjectManager. }  The object manager base class is defined in \text
{ riobjects.h. }  Three predefined object managers are provided by
\gman in the default implementation, Linear, BSP, and Octree.  The
\text{ ObjectManager } interface provides a generalized method for
accessing all objects in a World, i.e., all objects to be projected
into the output view by the shader.

\subsection{Shaders}
\gman also provides a generalized interface to shading algorithims.
These shading algorithims, or Shaders for short, are implemented
through an abstract base class, \text { Shader, } defined in \text
{ rishader.h. }  Any class derived from a \text { Shader } object can
be specified as the shader used in a particular GMAN rendering
session.  This allows for extreme flexiblity in implementation, and
reuse of \gman code, even where a different rendering algorithim is
desired.  This also provides a mechanism where under the LGPL, shaders
can be created in a client application that do not fall under the
terms of the license, as they need not be added to the \gman library,
they can be specified as part of the client application and integrated
into \gman at runtime.

\chapter{Shading Language Engine, \gmansl}
\section{Introduction}

The shading language engine is the core of the \gman shading
process.  The engine has two key components the \gmansl virtual machine
and the \gmansl compiler.  The \gmansl compiler parses the shading language
statements and grammars and produces a \gmansl code listing suitable for
loading and execution in the \gmansl virtual computer.

\section{The \gmansl Computer}

The \gmansl computer simulates a stack based computer specifically
designed, for rendering applications.  It behaves like a fairly simple
CISC machine.   It has a variable length stack, which allows operands
to be pushed and poped.  \gmansl provides just enough instructions for
the task at hand.

\subsection{General Description of the \gmansl Computer}

The \gmansl computer is a thirtytwo (32) bit computer.  In other words
the fundamental word size is four (4) bytes.  All values stored on the
\gmansl stack are four bytes long.  This is because the \gmansl
computer is required to run efficiently on thirty-two (32) bit
computers.  The \gmansl stack is 64k providing up to 16384 storage
locations. 

The \gmansl computer has 64k of random access storage accessible
through a single sixteen (16) bit index register, indx.  Addresses
must always be aligned on four (4) byte boundaries, as the low order 4
bits of the index register are ignored.

The \gmansl computer provides a 64k bank of storage shared with the
host computer, in particular this storage can be used for accessing
input and ouput storage for use with communicating with \gman.

\subsection{The \gmansl Instruction Set}

\instruction{ADD}{0x01}{The ADD instruction replaces the two
thirty-two bit floating point values on the stack with the sum of
these two values.}
\instruction{SUB}{0x02}{The SUB instruction replaces the two
thirty-two bit floating point values on the stack with the difference
of the stack top and the previous stack element.}


\end{document}
